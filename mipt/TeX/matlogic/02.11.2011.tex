\input{header.tex}

\begin{document}

Теорема:

  Если Q выразим в интерпретации < M; P1, \ldots , Pk; f1, \ldots , fk > ,
  а R выразим и в интерпретации <M; P1, \ldots, Pk, f1, \ldots fk >
  
  Идея доказательства: Заменим Q на формулу, его выражающую.
  Упражнение: Записать строгую формулировку и доказательство.

Если ${S_m(k)}$ = 00\ldots 0, то приписывание ${S_m(k)}$ соответсвует умножению на k+1
В общем случае - умножение на максимальную степень двойки ${\leq}$ k+1 и прибавление числа, для которого ${S_m(k)}$

\end{document}
